\documentclass[a4paper, 12pt]{scrartcl}
\usepackage[utf8]{inputenc}
\usepackage[T1]{fontenc}
\usepackage[ngerman]{babel}
\pagestyle{plain}
\title{Satzung}
\subtitle{des shack e.V.}
\author{}
\date{\sffamily von 2010-02-20\\\small Zuletzt geändert durch Beschluss von 2016-12-15}

\renewcommand*\thesection{\S~\arabic{section}}
% \renewcommand\labelenumii{\theenumi.\arabic{enumii}.}
\renewcommand\labelenumii{\arabic{enumii}.}

\KOMAoptions{toc=flat}
\addtokomafont{pagenumber}{\sffamily}
\begin{document}
\maketitle
\sffamily
% \tableofcontents

%\newpage
\section*{Präambel}
Die Informationsgesellschaft unserer Tage ist ohne Computer (dazu gehören auch alle Arten von mobilen oder embedded Systemen) und Internet nicht mehr denkbar. Mit steigender Komplexität der Technologien steigen die Anforderungen an den Nutzer sich in einer durch Informationstechnologie und ihrer Prozesse verändernden Zeit selbstbestimmt und kompetent zu bewegen.


\section{Name, Sitz, Geschäftsjahr}
     \begin{enumerate}
	  \item Der Verein führt den Namen ”shack”. Der Verein wird in das Vereinsregister eingetragen und dann um den Zusatz ”e.V.” ergänzt.  
	  \item Der Verein hat seinen Sitz in Stuttgart.  
	  \item Das Geschäftsjahr beginnt am 1. Januar jeden Kalenderjahres.
     \end{enumerate}

\section{Zweck und Gemeinnützigkeit}

     \begin{enumerate}
	  \item Der Verein ist politisch und religiös neutral und Dritten gegenüber ungebunden.
	  \item Der Verein verfolgt ausschliesslich und unmittelbar gemeinnützige Zwecke im Sinne des Abschnitts ”Steuerbegünstigte Zwecke” der Abgabenordnung.
	  \item Zwecke des Vereins sind die Förderung von Informatik- und Medienkompetenz von Jugendlichen und Erwachsenen, sowie die Aufklärung über Techniken, Risiken, Gefahren und Möglichkeiten der Medien, sowie die Wahrung der Menschenrechte. Darüber hinaus ist er ein Forum für künstlerisch-kritischen Umgang mit Medien- und Computerkultur. Durch die genannten Zwecke soll die Volks-, Jugend- und Erwachsenenbildung gefördert werden.  
  
	  \item Der Satzungszweck wird insbesondere verwirklicht durch:  
	       \begin{enumerate}
		    \item Informationsveranstaltungen, Öffentlichkeitsarbeit, Arbeitskreise, sowie Förderung des schöpferisch-kritischen Umgangs mit Informations- und Kommunikationstechnologien.  
		    \item Vorbereitung, Durchführung oder Förderung von Bildungs- und Kulturveranstaltungen (Kurse, Seminare, Workshops usw.) zur allgemeinen Erwachsenen- und Berufsbildung.  
		    \item Jugendbetreuung und -förderung durch Vermittlung von Kenntnissen und Fertigkeiten aus dem Bereich der Informations- und Kommunikationstechnik im Rahmen von Gruppenarbeit. Insbesondere sollen Kooperationen mit anderen Jugendvereinen in der Region Stuttgart angestrebt werden.  
		    \item Dialog und Kooperation mit technischen und kulturellen Einrichtungen vor allem der Früherziehung, Bildung, Weiterbildung und Praxis.  
	       \end{enumerate}
  
	  \item Die Körperschaft ist selbstlos tätig. Sie verfolgt nicht in erster Linie eigenwirtschaftliche Zwecke. Mittel der Körperschaft dürfen nur für die satzungsgemäßen Zwecke verwendet werden. Die Mitglieder erhalten keine Zuwendungen aus Mitteln der Körperschaft. Es darf keine Person durch Ausgaben, die dem Zweck der Körperschaft fremd sind oder durch unverhältnismäßig hohe Vergütung begünstigt werden. Jeder Beschluss über die Änderung der Satzung ist vor dessen Anmeldung beim Registergericht dem zuständigen Finanzamt vorzulegen.
     \end{enumerate}

\section{Mitgliedschaft}
     \begin{enumerate}
	  \item Ordentliche Vereinsmitglieder können natürliche und juristische Personen, Handelsgesellschaften, rechtsfähige Vereine sowie Anstalten und Körperschaften des öffentlichen Rechts werden.  
  
	  \item Die Beitrittserklärung erfolgt schriftlich oder fernschriftlich gegenüber dem Vorstand. Über die Annahme der Beitrittserklärung entscheidet der Vorstand. Die Mitgliedschaft beginnt mit der Annahme der Beitrittserklärung.  
  
	  \item Die Mitgliedschaft endet durch:  
	       \begin{enumerate}
		    \item Austritt  
		    \item den Tod von natürlichen Personen  
		    \item Auflösung und Erlöschung von juristischen Personen  
		    \item Ausschluss
	       \end{enumerate}
  
	  \item Der Austritt erfolgt unter Einhaltung einer Kündigungsfrist gemäß Beitragsordnung durch schriftliche oder fernschriftliche Willenserklärung gegenüber dem Vorstand.

	  \item Neben der aktiven (regulären) Mitgliedschaft ist eine passive Mitgliedschaft möglich.  
Diese entspricht der regulären Mitgliedschaft mit der Ausnahme, dass passive Mitglieder kein Stimmrecht inne haben. Andere Rechte und Pflichten bleiben hiervon unberührt.  
     \end{enumerate}

\section{Rechte und Pflichten der Mitglieder}
     \begin{enumerate}
	  \item Die Mitglieder sind berechtigt, die Leistungen des Vereins in Anspruch zunehmen.  
	  \item Die Mitglieder sind verpflichtet, die satzungsgemäßen Zwecke des Vereins zu unterstützen und zu fördern. Sie sind verpflichtet, die festgesetzten Beiträge zu zahlen.
	  \item Ein Beitragsrückstand von drei (3) Monaten führt automatisch zum Wechsel der Mitgliedschaft in den Status der passiven Mitgliedschaft.  
	  \item Mitglieder die durch Beitragsrückstand in die passiven Mitgliedschaft gewechselt wurden, werden nach Begleichen der ausstehenden Beitragszahlungen wieder als reguläre Mitglieder geführt
     \end{enumerate}

\section{Ausschluss eines Mitglieds}
     \begin{enumerate}
	  \item Ein Mitglied kann durch Beschluss des Vorstandes ausgeschlossen werden, wenn es das Ansehen des Vereins schädigt, seinen Beitragsverpflichtungen nicht nachkommt oder wenn ein sonstiger wichtiger Grund vorliegt. Der Vorstand muss dem auszuschließenden Mitglied den Beschluss in schriftlicher Form unter Angabe von Gründen mitteilen und ihm auf Verlangen eine Anhörung gewähren.  
	  \item Gegen den Beschluss des Vorstandes ist die Anrufung der Mitgliederversammlung zulässig. Bis zum Beschluss der Mitgliederversammlung ruht die Mitgliedschaft.
     \end{enumerate}

\section{Beitrag}
     \begin{enumerate}
	  \item Der Verein erhebt Beiträge. Das nähere regelt eine Beitragsordnung, die von der Mitgliederversammlung beschlossen wird.  
	  \item Im begründeten Einzelfall kann für ein Mitglied durch Vorstandsbeschluss ein von der Beitragsordnung abweichender Beitrag festgesetzt werden.
     \end{enumerate}

\section{Organe des Vereins}
     \begin{enumerate}
	  \item Die Mitgliederversammlung 
	  \item Der Vorstand
	  \item Der wissenschaftliche Beirat
     \end{enumerate}

\section{Mitgliederversammlung}
     \begin{enumerate}
	  \item Oberstes Beschlussorgan ist die Mitgliederversammlung. Ihrer Beschlussfassung unterliegen:
	       \begin{enumerate}
		    \item Die Genehmigung des Finanzberichtes,
		    \item Die Entlastung des Vorstandes,
		    \item Die Wahl der einzelnen Vorstandsmitglieder,
		    \item Die Bestellung von Finanzprüfern,
		    \item Satzungsänderungen,
		    \item Die Genehmigung der Beitragsordnung,
		    \item Die Richtlinie über die Erstattung von Reisekosten und Auslagen, 
		    \item Anträge des Vorstandes und der Mitglieder,
		    \item Die Ernennung von Ehrenmitgliedern,
		    \item Die Auflösung des Vereins.
	       \end{enumerate}
	  \item Die ordentliche Mitgliederversammlung findet alle zwei Jahre statt. Außerordentliche Mitgliederversammlungen werden auf Beschluss des Vorstandes abgehalten, wenn die Interessen des Vereins dies erfordern, oder wenn mindestens zehn Mitglieder dies unter Angabe des Zwecks schriftlich oder fernschriftlich beantragen. Die Einberufung der Mitgliederversammlung erfolgt schriftlich oder fernschriftlich durch den Vorstand mit einer Frist von mindestens zwei Wochen. Hierbei sind die Tagesordnung bekanntzugeben und die nötigen Informationen zugänglich zu machen. Anträge zur Tagesordnung sind mindestens drei Tage vor der Mitgliederversammlung bei der Geschäftsstelle einzureichen. Über die Behandlung von Initiaitvanträgen entscheidet die Mitgliederversammlung.

	  \item Die Mitgliederversammlung ist beschlussfähig, wenn mindestens fünfzehn Prozent aller Mitglieder anwesend oder durch Vollmacht vertreten sind. Beschlüsse sind jedoch gültig, wenn die Beschlussfähigkeit vor der Beschlussfassung nicht angezweifelt worden ist.
	  \item  Beschlüsse über Satzungsänderungen und über die Auflösung des Vereins bedürfen zu ihrer Rechtswirksamkeit der Dreiviertelmehrheit der anwesenden Mitglieder bzw. durch Vollmacht vertretenen Mitglieder. In allen anderen Fällen genügt die einfache Mehrheit.
	  \item Jedes Mitglied hat eine Stimme. Juristische Personen haben einen Stimmberechtigten schriftlich zu bestellen.

	  \item Jedes ordentliche Mitglied kann sich durch ein anderes, anwesendes ordentliches Mitglied vertreten lassen. Jedes anwesende ordentliche Mitglied kann, zusätzlich zu seiner eigenen Stimme, die Stimme maximal eines weiteren ordentlichen Mitglieds in Vertretung übernehmen. Die Vollmacht bedarf der Schriftform und muss dem Versammlungsleiter übergeben werden. Eine Einschränkung der Vollmacht durch den Bevollmächtigenden ist nicht möglich.
	  \item Auf Antrag eines Mitglieds ist geheim abzustimmen. Über die Beschlüsse der Mitgliederversammlung ist ein Protokoll anzufertigen, das vom Versammlungsleiter und dem Protokollführer zu unterzeichnen ist. Das Protokoll ist allen Mitgliedern zugänglich zu machen und auf der nächsten Mitgliederversammlung genehmigen zu lassen.
     \end{enumerate}

\section{Vorstand}
     \begin{enumerate}
	  \item Der Vorstand besteht aus einem Vorsitzenden (“1. Vorstand”), einem Schatzmeister (“Kassenwart”) und einem Schriftführer. Es können zwei stellvertretende Vorsitzende (“stellvertretender Vorstand”) und zwei weitere Beisitzer gewählt werden.  
	  \item In den Vorstand dürfen nur natürliche Personen gewählt werden.  
	  \item Der Verein wird gerichtlich und aussergerichtlich durch den Vorsitzenden oder durch zwei Mitglieder des Vorstandes gemeinsam vertreten.  
	  \item Sind mehr als zwei Vorstandsmitglieder dauernd an der Ausübung ihres Amtes gehindert, so sind unverzüglich Nachwahlen anzuberaumen  
	  \item Die Amtsdauer der Vorstandsmitglieder beträgt zwei Jahre. Wiederwahl ist zulässig.  
	  \item Jedes Vorstandsmitglied ist alleinvertretungsberechtigt im Sinne des §26, BGB bei Rechtsgeschäften bis zu einem Höchstbetrag von 1500 EURO. Bei Rechtsgeschäften über 1500 EURO ist die Vertretung durch zwei Vorstandsmitglieder erforderlich.  
	  \item Der Schatzmeister überwacht die Haushaltsführung und verwaltet das Vermögen des Vereins. Er hat auf eine sparsame und wirtschaftliche Haushaltsführung hinzuwirken. Mit dem Ablauf des Geschäftsjahres stellt er unverzüglich die Abrechnung, sowie die Vermögensübersicht und sonstige Unterlagen von wirtschaftlichem Belangen den Finanzprüfern des Vereins zur Prüfung zur Verfügung.  
	  \item Die Vorstandsmitglieder sind grundsätzlich ehrenamtlich tätig. Sie haben Anspruch auf Erstattung notwendiger Auslagen im Rahmen einer von der Mitgliederversammlung zu beschliessenden Richtlinie über die Erstattung von Reisekosten und Auslagen.  
     \end{enumerate}

\section{Der wissenschaftliche Beirat}
     \begin{enumerate}
	  \item Der Vorstand kann einen ”Wissenschaftlichen Beirat” einrichten, der für den Verein beratend und unterstützend tätig wird.  
	  \item In den Beirat können auch Nichtmitglieder berufen werden.
     \end{enumerate}

\section{Finanzprüfer}
     \begin{enumerate}
	  \item Zur Kontrolle der Haushaltsführung bestellt die Mitgliederversammlung Finanzprüfer. Nach Durchführung ihrer Prüfung setzen sie den Vorstand von ihrem Prüfungsergebnis in Kenntnis und erstatten der Mitgliederversammlung Bericht.  
	  \item Die Finanzprüfer dürfen dem Vorstand nicht angehören.
     \end{enumerate}


\section{Auflösung des Vereins}
Bei  der Auflösung  oder Aufhebung des Vereins oder bei Wegfall steuerbegünstigter Zwecke fällt das Vermögen des Vereins an den - ”Entropia e.V.”, Steinstrasse 23, 76133 Karlsruhe - der es unmittelbar und ausschließlich für gemeinnützige oder mildtätige Zwecke zu verwenden hat.      
\end{document}
